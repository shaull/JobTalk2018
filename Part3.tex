\begin{frame}
	\frametitle{Reasoning about Quality(1)}
	Different ways of satisfying a specification should have different levels of quality.
	
	\onslide<2->{Consider the specification ``breakfast'':}
		\begin{columns}
			\begin{column}{0.45\textwidth}
				\begin{overprint}	
					\onslide<3| handout:0>
					\begin{figure}[tbp]
						\includegraphics[ height=3.5cm]{graphics/egg1.pdf}
					\end{figure}
					
					\onslide<4| handout:0>
					\begin{figure}[tbp]
						\includegraphics[ height=3.5cm]{graphics/egg2.pdf}
					\end{figure}
					
					\onslide<5| handout:0>
					\begin{figure}[tbp]
						\includegraphics[ height=3.5cm]{graphics/egg3.pdf}
					\end{figure}
				\end{overprint}	
			\end{column}
			\begin{column}{0.45\textwidth}
				\begin{overprint}
					
					\onslide<3| handout:0>
					Poor quality.		
					\onslide<4| handout:0>
					Better quality.
					\onslide<5| handout:0>
					Om nom nom!
					
				\end{overprint}	
			\end{column}
		\end{columns}
\end{frame}


\begin{frame}
	\frametitle{Reasoning about Quality (2)}
	\begin{itemize}
		\item<+-> We add \emc{darkgreen}{quality} operators to LTL (e.g., $\min$, $\max$, $\oplus_\lambda$, $\factorU_\lambda$),
		\item<+-> as well as \emc{darkgreen}{discounting} operators.
		\item<+-> A formula $\phi$ has a satisfaction \emc{darkgreen}{value} in $[0,1]$.
		\item<+-> Intuitively, the higher the value is, the ``more satisfied'' the formula is.
		\item<+-> The system remains Boolean.
		\uncover<+->{
		\item Based on the following works:\\
		 {\footnotesize S. Almagor, U. Boker, and O. Kupferman}
		 
		\begin{itemize}
			\item[-] {\footnotesize Formalizing and Reasoning About Quality \textbf{ICALP 2013}.}
			\item[-] {\footnotesize Discounting in LTL \textbf{TACAS 2014}.}
			\item[-] {\footnotesize Formally Reasoning About Quality \textbf{J.ACM 2016}.}			
		\end{itemize}	}
	\end{itemize}
\end{frame}

\begin{frame}
	\frametitle{Example - Propositional Quality}
	\begin{itemize}[<+->]
		\item Consider a system that grants locks for a data structure.   
		\item If we get a \emc{blue}{write lock}, we are  happy (value {\color{blue} $1$}).~\includegraphics[height=12pt]{graphics/smileyHappy2.png}
		\item If we get a \emc{darkgreen}{read lock}, we are ambivalent (value {\color{darkgreen}$\frac{1}{2}$}).~\includegraphics[ height=12pt]{graphics/smileyAmb2.png}
		\item Otherwise we are not happy (value 0).~\includegraphics[height=12pt]{graphics/smileySad2.png}
		\item Formulated as:
		$$\Alw \Big(req\to \Next \emc{blue}{write} \vee \emc{darkgreen}{\factorU_{\frac12} \Next read})$$
	\end{itemize}
\end{frame}

\begin{frame}
	\frametitle{Example - Discounting the Future}
	\begin{itemize}[<+->]
		\item Consider a system that grants requests. A grant should be given \emc{blue}{as soon as possible}.
		\item We penalize the satisfaction \emc{darkgreen}{value} according to the \emc{red}{time} it takes for a grant to be given.
		\item We formulate the \emc{darkgreen}{quality} of the system as
		$$\Alw(req\to \emc{darkgreen}{\Ev_{0.9}} grant)$$
		\item[-] $\emc{darkgreen}{\Ev_{0.9}}$ means that if a grant is received after \emc{red}{$k$} steps, the satisfaction value is $\emc{darkgreen}{0.9}^{\emc{red}{k}}$,\\
		\item[-] $\Alw$ means that we take the minimal satisfaction value among prefixes.
		%\item $\emc{blue}{(req,(\neg grant)^3,grant)(req,(\neg grant)^2,grant)^\omega}$ gets value $\emc{blue}{0.9^3}$. 
		\item $\emc{blue}{(req,\emptyset,\emptyset,\emptyset,grant)(req,\emptyset,\emptyset,grant)^\omega}$ gets value $\emc{blue}{0.9^3}$. 
		%\item $\emc{red}{(req,(\neg grant^i),grant)_{i=0}^\infty}$ gets value $\emc{red}{0}$.
	\end{itemize}
\end{frame}

\begin{frame}
	\frametitle{Reasoning about Quality (3)}
	Results [\emph{ICALP 2013, TACAS 2014, and J.ACM 2016}]:
	\begin{itemize}[<+->]
		\item $LTL[{\cal F}]$ - $LTL$ augmented with \emc{darkgreen}{quality operators}:
			\begin{itemize}
				\item Model Checking and Satisfiability are PSPACE-Complete.
				\item Synthesis is 2EXPTIME-Complete.
			\end{itemize}
		\item $LTL[{\cal D}]$ - $LTL$ augmented with \emc{darkgreen}{discounting operators}:
			\begin{itemize}
				\item Strict-threshold Model Checking is PSPACE-Complete. (Requires PosSLP$\in$PSPACE).
				\item Non-strict threshold - open.
				\item Synthesis - open (can do approximate synthesis).
			\end{itemize}		
		\item $LTL[{\cal F+D}]$
		\begin{itemize}
			\item Already Satisfiability is \emc{red}{undecidable}.
			\item Some decidable fragments are known.
		\end{itemize}
	\end{itemize}
\end{frame}