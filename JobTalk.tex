\documentclass{beamer}
\usepackage{amssymb}
\usepackage{pifont}
\usepackage{amsmath}
\usepackage{wasysym}
\usepackage{xcolor}
\usepackage{bbm}
\usepackage{graphicx}
\usepackage{varwidth}
\usepackage{animate}
\usepackage{tikz}
\usetikzlibrary{positioning,calc}

\newcommand{\set}[1]{\left\{#1\right\}}
\newcommand{\NN}{{\mathbbm{N}}}
\newcommand{\ZZ}{{\mathbbm{Z}}}
\newcommand{\RR}{{\mathbbm{R}}}
\newcommand{\CC}{{\mathbbm{C}}}
\newcommand{\QQ}{{\mathbbm{Q}}}
\renewcommand{\AA}{{\mathbbm{A}}}


\newcommand{\B}{{\cal B}}
\newcommand{\A}{{\cal A}}
\renewcommand{\P}{{\cal P}}
\newcommand{\Q}{{\cal Q}}
\newcommand{\T}{{\cal T}}
\newcommand{\Inv}{{\cal I}}

\newcommand{\Next}{\mathsf{X}}
\newcommand{\Ev}{\mathsf{F}}
\newcommand{\Alw}{\mathsf{G}}
\newcommand{\Until}{\mathsf{U}}
\newcommand{\Release}{\mathsf{R}}

\renewcommand{\phi}{\varphi}

\newcommand{\factorU}{\triangledown}

\newcommand{\sen}{{\it sensed}}
\newcommand{\scost}{{\it scost}}

\newcommand{\stam}[1]{}

%COLORS

\definecolor{darkgreen}{RGB}{20,149,50}
\definecolor{darkblue}{RGB}{20,50,149}
\definecolor{orange}{RGB}{200,100,0}
\definecolor{lemmacol}{RGB}{30,140,80}
\definecolor{bgc}{RGB}{255,255,255}

\definecolor{specc}{RGB}{20,149,50}

\definecolor{asc}{RGB}{117,50,255}
\definecolor{wc}{RGB}{0,220,255}
\definecolor{stochc}{RGB}{250,0,133}

\definecolor{sensec}{RGB}{20,50,170}
\definecolor{envc}{RGB}{149,50,20}
\definecolor{probc}{RGB}{50,170,130}
\definecolor{targetc}{RGB}{20,149,50}
\definecolor{initc}{RGB}{20,50,149}
\definecolor{lcostc}{RGB}{65,140,160}
\definecolor{wcostc}{RGB}{140,65,160}

\newcommand{\emc}[2]{{\color{#1} #2}}

\newcommand{\cmark}{{\color{darkgreen} \ding{51}}}%
\newcommand{\xmark}{{\color{red}  \ding{55}}}%

\newcommand{\colInit}[1]{\emc{initc}{#1}}
\newcommand{\colTarget}[1]{\emc{targetc}{#1}}

\newcommand{\conj}[1]{{\overline{#1}}}
\newcommand{\norm}[1]{{\left\lVert#1\right\rVert}}

\newcommand{\relBow}{\mathrel{\bowtie}}

\newcommand{\itemnow}{{\raisebox{-4pt}{\includegraphics[scale=0.5]{graphics/Now1.png}}}}
\newcommand{\itemlater}{{\raisebox{-6pt}{\includegraphics[scale=0.47]{graphics/Later1.png}}}}

%-- beamer funcs --
\newcommand{\semitransp}[2][35]{\color{fg!#1}#2}
\newenvironment<>{grayenv}{%
  \color#1{gray}}{}

\def\colorize<#1>{%
  \temporal<#1>{\color{black!50}}{\color{black}}{\color{black!50}}}

\setbeamercolor{uppercolgreen}{fg=black,bg=green!45}%
\setbeamercolor{lowercolgreen}{fg=black,bg=green!5}%
\setbeamercolor{uppercolred}{fg=black,bg=red!35}% 
\setbeamercolor{lowercolred}{fg=black,bg=red!10}% 
\setbeamercolor{uppercoldarkred}{fg=black,bg=red!55}% 
\setbeamercolor{lowercoldarkred}{fg=black,bg=red!20}% 
\setbeamercolor{uppercolblue}{fg=white,bg=blue!35}% 
\setbeamercolor{lowercolblue}{fg=black,bg=blue!10}% 
\setbeamercolor{uppercollem}{fg=white,bg=lemmacol!60}% 
\setbeamercolor{lowercollem}{fg=black,bg=lemmacol!10}% 


\definecolor{darkgreen}{RGB}{20,149,50}
\definecolor{orange}{RGB}{200,100,0}
\definecolor{lemmacol}{RGB}{30,140,80}
\definecolor{proofcol}{RGB}{30,140,80}
\definecolor{bgc}{RGB}{255,255,255}
\definecolor{light}{RGB}{180,180,180}



\definecolor{sizec}{RGB}{20,149,50}
\definecolor{sensec}{RGB}{20,50,170}
\definecolor{limitc}{RGB}{149,50,20}
\definecolor{probc}{RGB}{50,170,130}
\definecolor{ic}{RGB}{0,88,204}
\definecolor{oc}{RGB}{204,116,0}


\setbeamercolor{background canvas}{bg=bgc}

%-- end of funcs  --



\mode<presentation>
{ \usetheme{Singapore} } 
\setbeamertemplate{footline}[frame number]
\beamertemplatenavigationsymbolsempty
%\setbeamertemplate{sections/subsections in toc}[default]

%\setbeamertemplate{background}{\tikz[overlay, remember picture, help lines]{
%    \foreach \x in {0,...,12} \path (current page.south west) +(\x,8.25) node {\small$\x$};
%    \foreach \y in {0,...,9} \path (current page.south west) +(12.5,\y) node {\small$\y$};
%    \foreach \x in {0,0.5,...,12.5} \draw (current page.south west) ++(\x,0) -- +(0,9.6);
%    \foreach \y in {0,0.5,...,9.5} \draw (current page.south west) ++(0,\y) -- +(12.8,0);
%  }
%}



\title{Automated Verification of Infinite-State Systems}
\subtitle{with Certifiable Algorithms}
\author{Shaull Almagor}
\institute{Oxford University}
\date{2018}

\begin{document}

\setcounter{subsection}{1}

\begin{frame}
\titlepage
\end{frame}



\section{Model Checking}
\stepcounter{subsection}

\begin{frame}
\frametitle{Automated Verification - Traditional Approach}
\only<1>{\includegraphics[page=1, scale=0.4]{graphics/MC}}
\only<2>{\includegraphics[page=2, scale=0.4]{graphics/MC}}
\end{frame}

\begin{frame}
\frametitle{Automated Verification - Example}
\only<1-4>{\includegraphics[page=1, scale=1]{graphics/reqGrantSys}}
\only<5-7>{\includegraphics[page=2, scale=1]{graphics/reqGrantSys}}
\onslide<+->
\onslide<+->

Property: Every request is eventually followed by a grant.\\
\onslide<+->
In {\em Linear Temporal Logic} (LTL):\\
 $\Alw({req}\to \Ev\ grant)$. 
\only<4>{\cmark}
\only<6-7>{\xmark}

\onslide<7>{Counterexample: $\emc{darkgreen}{1,2,3,3,3^\omega}$}
\end{frame}


\begin{frame}
\frametitle{Automated Verification - Success Stories}
\begin{itemize}
\item \emc{red}{Turing Awards}:
\begin{itemize}
\item[-] \emph{A. Pnueli}, 1996. For introducing temporal logic.
\item[-] \emph{E.M. Clarke, E.A. Emerson}, and \emph{J. Sifakis}, 2007.\\ For developing model checking.
\end{itemize} 
\item \emc{red}{Kanellakis Theory and Practice Awards}:
\begin{itemize}
\item[-] \emph{R. Bryant, E.M. Clarke, E.A. Emerson}, and \emph{K.L. McMillan}, 1998. For symbolic model checking.
\item[-] \emph{G. Holzmann, R. Kurshan, M. Vardi}, and \emph{P. Wolper}, 2005.\\ For techniques in formal verification.
\end{itemize}
\item \emc{red}{ACM software award}: 
\begin{itemize}
\item[-] \emph{G. Holzmann}, 2001. For SPIN model checker.
\end{itemize}
\item Used extensively in practice: 
\item[] \includegraphics[scale=0.3]{graphics/hardware}
\item[] \includegraphics[scale=0.4]{graphics/software}
\end{itemize}
\end{frame}


\begin{frame}
\frametitle{Automated Verification - Limitations}
\begin{itemize}
\item[1. \xmark] \alt<1>{Systems must be concrete and finite state.}{\textbf{Systems must be concrete and finite state.}}
\begin{itemize}
\item[]
\end{itemize}
\item[2. \xmark] Specifications are Boolean - not rich enough.
\begin{itemize}
\item[]
\end{itemize}
\item[3. \xmark] What happens with ``yes'' instances?
\begin{itemize}
\item[] 
\end{itemize}
\end{itemize}
\pause
\end{frame}

\begin{frame}
\frametitle{Systems must be concrete and finite state}
\begin{itemize}[<+->]
\item Chip with 64 bits of memory \onslide<+->{$\to$ \emc{red}{$2^{64}$} states! }
\item Code with a counter \onslide<4->{$\to$ \emc{red}{Infinitely many} states!}
	\onslide<3->{\begin{beamerboxesrounded}[upper=uppercolblue,lower=lowercolblue,shadow=true, width=7cm]{Collatz$(n)$}
		\texttt{while} $n\neq 1$\\
		\hspace*{.5cm} 	\texttt{if} $n$ is even, \texttt{then} $n\leftarrow n/2$\\
		\hspace*{.5cm} 	\texttt{else if} $n$ is odd, \texttt{then} $n\leftarrow 3n+1$
	\end{beamerboxesrounded}}
\onslide<5->
\item Robot in an arena \onslide<6->{$\to$ \emc{red}{Continuous} state space!}

\includegraphics[scale=0.3]{graphics/RobotPlan.png}
\end{itemize}
\end{frame}

\begin{frame}
\frametitle{Automated Verification - Limitations}
\begin{itemize}
\item[1. \xmark] Systems must be concrete and finite state.
\begin{itemize}
\item[]
\end{itemize}
\item[2. \xmark] \alt<1>{Specifications are Boolean - not rich enough.}{\textbf{Specifications are Boolean - not rich enough.}}
\begin{itemize}
\item[]
\end{itemize}
\item[3. \xmark] What happens with ``yes'' instances?
\begin{itemize}
\item[] 
\end{itemize}
\end{itemize}
\pause
\end{frame}

\begin{frame}
\frametitle{Specifications are Boolean - not rich enough}
\begin{itemize}
\item Boolean properties are not expressive enough for today's systems:
\begin{itemize}
\item \emc{red}{How many} packages are lost?
\item What is the \emc{red}{worst-case waiting time} for a lock?
\item \emc{red}{Minimize} the average energy consumption.
\item Stay as \emc{red}{far away} as possible from ``bad'' regions.
\end{itemize}
\end{itemize}
\end{frame}

\begin{frame}
\frametitle{Automated Verification - Limitations}
\begin{itemize}
\item[1. \xmark] Systems must be concrete and finite state.
\begin{itemize}
\item[]
\end{itemize}
\item[2. \xmark] Specifications are Boolean - not rich enough.
\begin{itemize}
\item[]
\end{itemize}
\item[3. \xmark] \alt<1>{What happens with ``yes'' instances?}{\textbf{What happens with ``yes'' instances?}}
\begin{itemize}
\item[] 
\end{itemize}
\end{itemize}
\pause
\pause
\begin{tikzpicture}[overlay]
\node at ($(current page.south west)+(0.8,1.2)$) [anchor=north west] {\includegraphics[scale=0.39]{graphics/grandparents.jpg}};
\node at ($(current page.south west)+(5.5,0.6)$) [anchor=north west, align=left] {\emph{``Convince not only the designer,} \\ \hspace*{0.1cm} \emph{but the designer's grandparents''}\\ \hspace*{3.1cm}- J. Ouaknine
 };
\end{tikzpicture}

\end{frame}

\section{Beyond Boolean}
\stepcounter{subsection}
\begin{frame}
\frametitle{Beyond the Boolean Setting}
\hspace*{2.8cm}
\begin{itemize}
\item[1. \xmark] Systems must be concrete and finite state.
\begin{itemize}
\item[\cmark]<2-> Then reason about infinite-state systems.
\end{itemize}
\item[2. \xmark] Specifications are Boolean - not rich enough.
\begin{itemize}
\item[\cmark]<3-> Then allow non-Boolean specifications.
\end{itemize}
\item[3. \xmark] What happens with ``yes'' instances?
\begin{itemize}
\item[\cmark]<4-> Output some proof/witness of correctness.
\end{itemize}
\onslide<5->{
\begin{tikzpicture}[overlay]
\node at ($(current page.south west)+(-0.3,4.3)$) [anchor=north west] {\includegraphics[scale=0.3]{graphics/doneTransp.png}};
\end{tikzpicture}
}
\end{itemize}
\end{frame}

\begin{frame}
 \frametitle{Fantastic Systems {\normalsize and How to Reason About Them}}
Are we done?
\onslide<+->
{\footnotesize
\begin{itemize}[<+->]
\item Systems that are too rich do not admit decidable/tractable model-checking, even for simple specifications.
\begin{itemize}
	\item e.g., Turing Machines.
\end{itemize}
\item Specifications that are too rich do not admit decidable/tractable model-checking, even for simple systems.
\begin{itemize}
	\item e.g., Peano Arithmetic.
\end{itemize}
\item It is not always clear what a proof/witness means, \\nor whether we can find one. 
\end{itemize}
}
\end{frame}

\begin{frame}
	\frametitle{Fantastic Systems {\normalsize and How to Reason About Them}}
	\begin{tikzpicture}[overlay]
	\node at ($(current page.south west)+(-0.5,3.2)$) [anchor=north west] {\includegraphics[scale=0.31]{graphics/complexity2.png}};
	\end{tikzpicture} 	
 \end{frame}

\begin{frame}
\frametitle{Talk Outline}
\begin{itemize}
\item Non-Boolean Reasoning:
\begin{itemize} 
    \item[\itemnow] \textbf{\emc{red}{Reasoning about Quality - quantitative specifications.}}
	\item[\itemlater] Regular Sensing - a new complexity measure for $\omega$-regular languages.
\end{itemize}
\item Linear Dynamical Systems: 
\begin{itemize}
\item[\itemlater] The Polytope Collision Problem - input quantification in infinite-state systems.
\item[\itemlater] Semialgebraic invariant synthesis - certifiable algorithms for orbit problems.
\end{itemize}
\end{itemize}
\end{frame}

%\section{Reasoning about Quality}
%\stepcounter{subsection}
\begin{frame}
	\frametitle{Reasoning about Quality(1)}
	Different ways of satisfying a specification should have different levels of quality.
	
	\onslide<2->{Consider the specification ``breakfast'':}
		\begin{columns}
			\begin{column}{0.45\textwidth}
				\begin{overprint}	
					\onslide<3| handout:0>
					\begin{figure}[tbp]
						\includegraphics[ height=3.5cm]{graphics/egg1.pdf}
					\end{figure}
					
					\onslide<4| handout:0>
					\begin{figure}[tbp]
						\includegraphics[ height=3.5cm]{graphics/egg2.pdf}
					\end{figure}
					
					\onslide<5| handout:0>
					\begin{figure}[tbp]
						\includegraphics[ height=3.5cm]{graphics/egg3.pdf}
					\end{figure}
				\end{overprint}	
			\end{column}
			\begin{column}{0.45\textwidth}
				\begin{overprint}
					
					\onslide<3| handout:0>
					Poor quality.		
					\onslide<4| handout:0>
					Better quality.
					\onslide<5| handout:0>
					Om nom nom!
					
				\end{overprint}	
			\end{column}
		\end{columns}
\end{frame}


\begin{frame}
	\frametitle{Reasoning about Quality (2)}
	\begin{itemize}
		\item<+-> We add \emc{darkgreen}{quality} operators to LTL (e.g., $\min$, $\max$, $\oplus_\lambda$, $\factorU_\lambda$),
		\item<+-> as well as \emc{darkgreen}{discounting} operators.
		\item<+-> A formula $\phi$ has a satisfaction \emc{darkgreen}{value} in $[0,1]$.
		\item<+-> Intuitively, the higher the value is, the ``more satisfied'' the formula is.
		\item<+-> The system remains Boolean.
		\uncover<+->{
		\item Based on the following works:\\
		 {\footnotesize S. Almagor, U. Boker, and O. Kupferman}
		 
		\begin{itemize}
			\item[-] {\footnotesize Formalizing and Reasoning About Quality \textbf{ICALP 2013}.}
			\item[-] {\footnotesize Discounting in LTL \textbf{TACAS 2014}.}
			\item[-] {\footnotesize Formally Reasoning About Quality \textbf{J.ACM 2016}.}			
		\end{itemize}	}
	\end{itemize}
\end{frame}

\begin{frame}
	\frametitle{Example - Propositional Quality}
	\begin{itemize}[<+->]
		\item Consider a system that grants locks for a data structure.   
		\item If we get a \emc{blue}{write lock}, we are  happy (value {\color{blue} $1$}).~\includegraphics[height=12pt]{graphics/smileyHappy2.png}
		\item If we get a \emc{darkgreen}{read lock}, we are ambivalent (value {\color{darkgreen}$\frac{1}{2}$}).~\includegraphics[ height=12pt]{graphics/smileyAmb2.png}
		\item Otherwise we are not happy (value 0).~\includegraphics[height=12pt]{graphics/smileySad2.png}
		\item Formulated as:
		$$\Alw \Big(req\to \Next \emc{blue}{write} \vee \emc{darkgreen}{\factorU_{\frac12} \Next read})$$
	\end{itemize}
\end{frame}

\begin{frame}
	\frametitle{Example - Discounting the Future}
	\begin{itemize}[<+->]
		\item Consider a system that grants requests. A grant should be given \emc{blue}{as soon as possible}.
		\item We penalize the satisfaction \emc{darkgreen}{value} according to the \emc{red}{time} it takes for a grant to be given.
		\item We formulate the \emc{darkgreen}{quality} of the system as
		$$\Alw(req\to \emc{darkgreen}{\Ev_{0.9}} grant)$$
		\item[-] $\emc{darkgreen}{\Ev_{0.9}}$ means that if a grant is received after \emc{red}{$k$} steps, the satisfaction value is $\emc{darkgreen}{0.9}^{\emc{red}{k}}$,\\
		\item[-] $\Alw$ means that we take the minimal satisfaction value among prefixes.
		%\item $\emc{blue}{(req,(\neg grant)^3,grant)(req,(\neg grant)^2,grant)^\omega}$ gets value $\emc{blue}{0.9^3}$. 
		\item $\emc{blue}{(req,\emptyset,\emptyset,\emptyset,grant)(req,\emptyset,\emptyset,grant)^\omega}$ gets value $\emc{blue}{0.9^3}$. 
		%\item $\emc{red}{(req,(\neg grant^i),grant)_{i=0}^\infty}$ gets value $\emc{red}{0}$.
	\end{itemize}
\end{frame}

\begin{frame}
	\frametitle{Reasoning about Quality (3)}
	Results [\emph{ICALP 2013, TACAS 2014, and J.ACM 2016}]:
	\begin{itemize}[<+->]
		\item $LTL[{\cal F}]$ - $LTL$ augmented with \emc{darkgreen}{quality operators}:
			\begin{itemize}
				\item Model Checking and Satisfiability are PSPACE-Complete.
				\item Synthesis is 2EXPTIME-Complete.
			\end{itemize}
		\item $LTL[{\cal D}]$ - $LTL$ augmented with \emc{darkgreen}{discounting operators}:
			\begin{itemize}
				\item Strict-threshold Model Checking is PSPACE-Complete. (Requires PosSLP$\in$PSPACE).
				\item Non-strict threshold - open.
				\item Synthesis - open (can do approximate synthesis).
			\end{itemize}		
		\item $LTL[{\cal F+D}]$
		\begin{itemize}
			\item Already Satisfiability is \emc{red}{undecidable}.
			\item Some decidable fragments are known.
		\end{itemize}
	\end{itemize}
\end{frame}

%\section{Sensing}
%\stepcounter{subsection}

\begin{frame}
\frametitle{Talk Outline}
\begin{itemize}
\item Non-Boolean Reasoning:
\begin{itemize} 
    \item \emc{gray}{Reasoning about Quality - quantitative specifications.}
	\item[\itemnow] \textbf{\emc{red}{Regular Sensing - a new complexity measure for $\omega$-regular languages.}}
\end{itemize}
\item Linear Dynamical Systems:
\begin{itemize}
\item[\itemlater] The Polytope Collision Problem - input quantification in infinite-state systems.
\item[\itemlater] Semialgebraic invariant synthesis - certifiable algorithms for orbit problems.
\end{itemize}
\end{itemize}
\end{frame}

\begin{frame}
	\frametitle{Regular Sensing}
	\begin{itemize}[<+->]
		\item A new \emc{darkgreen}{complexity} measure for regular languages.
		\item Consider an alphabet $2^P$ for some finite set $P$
		\item Consider a DFA $\A$ for a regular language $L$ over $2^P$.
		\item Reading each signal in $P$ requires activating a sensor.
		\item The complexity $\emc{sensec}{\scost(\A)}$ of $\A$ can be measured by the \emc{limitc}{expected} number of sensors $\A$ uses.
		\item The complexity of a language $L$ is $\emc{sensec}{\scost(L)}=\inf\set{\emc{sensec}{\scost(\A)}: L(\A)=L}$.
		
		\begin{beamerboxesrounded}[upper=uppercollem,lower=lowercollem,shadow=true]{Main problem:}
					For a language/specification $L$, compute $\emc{sensec}{\scost(L)}$.
				\end{beamerboxesrounded}
	\end{itemize}
\end{frame}

\begin{frame}
	\frametitle{Sensing Example (1)}
	Take $P=\set{a,b}$.
	\only<1>{\includegraphics[page=1]{graphics/SensingPics}} %\input{a3v0}}
	\only<2-3>{\includegraphics[page=2]{graphics/SensingPics}} %\input{a3v1}}
	\only<4-5>{\includegraphics[page=4]{graphics/SensingPics}} %\input{a3v2}}
	\only<6>{\includegraphics[page=6]{graphics/SensingPics.pdf}} %\includegraphics[page=2]{SensingPics.pdf}} %\input{a3v3}}
	\only<7>{\includegraphics[page=7]{graphics/SensingPics.pdf}} %\input{a3v4}}
	\onslide<+->
	\onslide<+->
	\begin{itemize}
		\item<+-> Reading $b$ in $q_0$ does not affect the transition. Reading $a$ does affect it. Thus, $\sen(q_0)=\set{a}$ and \emc{sensec}{$\scost(q_0)=1$}. \onslide<+->
		\item<+->  $\sen(q_1)=\set{a,b}$, thus \emc{sensec}{$\scost(q_1)=2$}. 
		\item<+->  $\sen(q_2)=\set{a}$, thus \emc{sensec}{$\scost(q_2)=1$}. 
	\end{itemize}
\end{frame}

\begin{frame}
	\frametitle{Sensing Example (2)}
	\only<1>{\includegraphics[page=16]{graphics/SensingPics.pdf}} %\input{a3v5}}
	\only<2>{\includegraphics[page=17]{graphics/SensingPics.pdf}} %\input{a3v6}}
	\only<3->{\includegraphics[page=18]{graphics/SensingPics.pdf}} %\input{a3v7}}
	\onslide<+->
	\begin{itemize}[<+->]
		\item Each transition is given \emc{probc}{probability} $\frac{1}{|2^P|}$.
		\item The \emc{limitc}{limiting distribution} of $\A$.
		\item $\emc{sensec}{\scost(\A)}=\emc{limitc}{\frac25} \cdot \emc{sensec}{1}+\emc{limitc}{\frac25}\cdot \emc{sensec}{2}+\emc{limitc}{\frac15}\cdot \emc{sensec}{1}=\frac75$.
	\end{itemize}
\end{frame}

\begin{frame}
	\frametitle{Regular Sensing - Results}
	\begin{itemize}
		\item<+-> For \emc{blue}{finite words} - minimal \emc{sensec}{sensing} is attained by the (unique) minimal-\emc{orange}{size} DFA.
		\item<+-> For \emc{purple}{infinite words} - minimal \emc{sensec}{sensing} may be attained only as a \emc{limitc}{limit} of a sequence of automata.
		\item<+-> For \emc{purple}{infinite words} - nevertheless, minimal \emc{sensec}{sensing} is computable in polynomial time.
		\item<+-> Extends to the setting of open systems (synthesis).
		\item<+-> Applications in Monitoring.
		\item<+-> Based on the following works:\\
		{\footnotesize S. Almagor, D. Kuperberg, and O. Kupferman
		\begin{itemize}
			\item[-] {\footnotesize Regular Sensing \textbf{FSTTCS 2014}.}
			\item[-] {\footnotesize The Sensing Cost of
				Monitoring and Synthesis \textbf{FSTTCS 2015}.}
			\item[-] {\footnotesize Sensing as a Complexity
				Measure \textbf{DCFS 2017}.}
		\end{itemize}
		{\footnotesize S. Almagor, O. Kupferman, and Y. Velner
		\begin{itemize}
			\item[-] {\footnotesize Minimizing Expected Cost	Under Hard Boolean Constraints with Applications to Quantitative Synthesis \textbf{CONCUR 2016}}	
		\end{itemize}}}
		
	\end{itemize}
\end{frame}

\section{Linear Dynamical Systems}
\stepcounter{subsection}
\begin{frame}
\frametitle{Talk Outline}
\begin{itemize}
\item \emc{gray}{Non-Boolean Reasoning:}
\begin{itemize} 
    \item \emc{gray}{Reasoning about Quality - quantitative specifications.}
	\item \emc{gray}{Regular Sensing - a new complexity measure for $\omega$-regular languages.}
\end{itemize}
\item[\itemnow] \textbf{\emc{red}{Linear Dynamical Systems:}}
\begin{itemize}
\item[\itemlater] The Polytope Collision Problem - input quantification in infinite-state systems.
\item[\itemlater] Semialgebraic invariant synthesis - certifiable algorithms for orbit problems.
\end{itemize}
\end{itemize}
\end{frame}


\begin{frame}
	\frametitle{Linear Dynamical Systems}
	\begin{itemize}
		\item We consider a Linear Dynamical System:
		\onslide<+->
		\begin{itemize}[<+->]
			\item Start with a vector $\colInit{\vec{x}}\in \RR^d$.
			\item Progress by applying a matrix $A\in \RR^{d\times d}$.
			\item Reason about the {\em Orbit} ${\cal O}=\set{A^n \colInit{\vec{x}}:n\in \NN}$.
		\end{itemize}
	\item<+-> Models an infinite state machine.\\
	\hspace*{2.5cm}
		\begin{beamerboxesrounded}[upper=lowercolblue,lower=lowercolblue,shadow=true, width=3cm]{}
			\texttt{init} $\colInit{\vec{x}}$\\
			\texttt{while}({\em condition})\\
			\hspace*{0.5cm} $\colInit{\vec{x}}\leftarrow A\colInit{\vec{x}}$
		\end{beamerboxesrounded}
	\onslide<+->
	\item Properties to consider:
%	{\footnotesize 
%			\onslide<+->
%		\item[-] Vector-Reachability: Given a vector $\colTarget{\vec{y}}\in \RR^d$, $\exists n\in \NN$ s.t. $A^n \colInit{\vec{x}}=\colTarget{\vec{y}}$?
%			\onslide<+->
%		\item[-] Set-Reachability: Given a target set $\colTarget{T}\subseteq \RR^d$, $\exists n\in \NN$ s.t. $A^n \colInit{\vec{x}}\in \colTarget{T}$?
%			\onslide<+->
%		\item[-] (Non)Safety: Given sets $\colInit{S},\colTarget{T}\subseteq \RR^d$, $\exists \colInit{\vec{x}}\in \colInit{S}, \exists n\in \NN$ s.t.  $A^n \colInit{\vec{x}}\in \colTarget{T}$?
%			\onslide<+->
%		\item[-] Termination: Given sets $\colInit{S},\colTarget{T}\subseteq \RR^d$, $\forall \colInit{\vec{x}}\in \colInit{S}, \exists n\in \NN$ s.t.  $A^n \colInit{\vec{x}}\in \colTarget{T}$?
%	}
	{\footnotesize 
		\setlength{\tabcolsep}{1pt}
		\begin{tabular}{l l l l}
		\onslide<+->
		Vector-Reachability: & Given a vector $\colTarget{\vec{y}}\in \RR^d$, & &  $\exists n\in \NN$ s.t. $A^n \colInit{\vec{x}}=\colTarget{\vec{y}}$? \\ 
		\onslide<+->
		 Set-Reachability: & Given a set $\colTarget{T}\subseteq \RR^d$, & & $\exists n\in \NN$ s.t. $A^n \colInit{\vec{x}}\in \colTarget{T}$? \\ 
		\onslide<+->		 
		 (Non)Safety: & Given sets $\colInit{S},\colTarget{T}\subseteq \RR^d$, &  $\exists \colInit{\vec{x}}\in \colInit{S}$, & $\exists n\in \NN$ s.t.  $A^n \colInit{\vec{x}}\in \colTarget{T}$?\\
		\onslide<+->		 
		 Termination: & Given sets $\colInit{S},\colTarget{T}\subseteq \RR^d$, & $\forall \colInit{\vec{x}}\in \colInit{S},$ & $\exists n\in \NN$ s.t.  $A^n \colInit{\vec{x}}\in \colTarget{T}$?
		\end{tabular} 
	}	
	\end{itemize}
\end{frame}

\begin{frame}
	\frametitle{Orbit Problems - History}
	\pause
	\onslide<2->{
		\begin{beamerboxesrounded}[upper=lowercolblue,lower=lowercolgreen,shadow=true]
			{Given vectors $\colInit{\vec{x}}, \colTarget{\vec{y}}\in \QQ^d$ and a matrix $A\in \QQ^{d\times d}$, decide whether there exists $n\in \NN$ such that $A^n\colInit{\vec{x}}=\colTarget{\vec{y}}$.}
			\onslide<3->{ 
				Shown to be decidable in {\bf P} by Kannan and Lipton in [STOC 1980, JACM 1986].
			}
		\end{beamerboxesrounded}
	}
	\onslide<4->{
		\begin{beamerboxesrounded}[upper=lowercolblue,lower=lowercolgreen,shadow=true]
			{Given a vector $\colInit{\vec{x}}\in \QQ^d$, a halfspace $\colTarget{V}$ and a matrix $A\in \QQ^{d\times d}$, decide whether there exists $n\in \NN$ such that $A^n\colInit{\vec{x}}\in \colTarget{V}$.}
			\onslide<7->{
				Hard-open$^*$ in general. 				
				Solvable when $\colTarget{V}$ has a low dimension. 
				
				[V. Chonev, J. Ouaknine, and J. Worrell. STOC 2013, J.ACM 2016]
			}
		\end{beamerboxesrounded}
	}
	\onslide<8->{
		\begin{beamerboxesrounded}[upper=lowercolblue,lower=lowercolgreen,shadow=true]
			{Given a vector $\colInit{\vec{x}}\in \QQ^d$, a polytope $\colTarget{\Q}$ and a matrix $A\in \QQ^{d\times d}$, decide whether there exists $n\in \NN$ such that $A^n\colInit{\vec{x}}\in \colTarget{\Q}$.}
			\onslide<9->{
				Hard-open$^*$ in general. 				
				Solvable when $\colTarget{\Q}$ has dimension $\le 3$. 
				
				[V. Chonev, J. Ouaknine, and J. Worrell. SODA 2015]
			}
		\end{beamerboxesrounded}	
	}
	\only<5>{\begin{tikzpicture}[overlay]
	\node at ($(current page.center)-(1,1)$) [anchor=center, rotate=10] {\includegraphics[scale=0.4]{graphics/newspaper.pdf}}; 
	\end{tikzpicture} 	}	
	\onslide<7->{\small * w.r.t. long-standing number-theoretic open problems (Diophantine approximation)}	
	
\end{frame}   

\begin{frame}
\frametitle{Orbits - What's the Big Deal? (I)} 
\begin{itemize}[<+->]
	\item Consider a matrix $A$ and a vector $\colInit{\vec{x}}$.
	\item Assume $A$ is diagonalizable, then we can write 
	\onslide<+->
	$$A=P^{-1}\begin{pmatrix}
	\lambda_1 & & \\
	& \ddots &\\
	& & \lambda_d
	\end{pmatrix}P\ \text{ with } \lambda_i\in \CC.$$ 
	\item Thus, every {\em coordinate} of $A^n\colInit{\vec{x}}$ is of the form
	$$a_1\lambda^n_1+\ldots+a_d\lambda^n_d.$$
	\item This is {\em dominated} by the maximal modulus $|\lambda_{\max}|$.
	\item Smaller eigenvalues have an {\em exponentially decreasing} effect.
\end{itemize}
\end{frame}

\begin{frame}
\frametitle{Orbits - What's the Big Deal? (II)}
\begin{tikzpicture}[remember picture, overlay]
\draw [->,thick] (current page.south west) +(6,0.3) -- +(6,8);
\draw [->,thick] (current page.south west) +(0.3,4) -- +(12,4);	
\pause
\node at ($(current page.south west)+(2,7.5)$) {If $|\lambda_{\max}|>1$};   
\pause
\node at ($(current page.south west)+(4.8,3)$) {$\colInit{\vec{x}}$};
\draw[fill] ($(current page.south west)+(5,3)$) circle [radius=0.06];
\pause
\node at ($(current page.south west)+(8.8,5)$) {$\colTarget{\vec{y}}$};
\draw[fill] ($(current page.south west)+(9,5)$) circle [radius=0.06];
\pause
\draw[fill] ($(current page.south west)+(5.5,2.6)$) circle [radius=0.06];
\pause
\draw[fill] ($(current page.south west)+(6.3,2.3)$) circle [radius=0.06];
\pause
\draw[fill] ($(current page.south west)+(7,3)$) circle [radius=0.06];
\pause
\draw[fill] ($(current page.south west)+(6.8,4.2)$) circle [radius=0.06];
\pause
\draw[fill] ($(current page.south west)+(5.7,4.8)$) circle [radius=0.06];
\pause
\draw[fill] ($(current page.south west)+(4,4.1)$) circle [radius=0.06];
\pause
\draw[fill] ($(current page.south west)+(4.1,2)$) circle [radius=0.06];
\pause
\draw[fill] ($(current page.south west)+(6.5,1.5)$) circle [radius=0.06];
\pause
\draw[fill] ($(current page.south west)+(8.3,2.5)$) circle [radius=0.06];
\pause
\draw[fill] ($(current page.south west)+(9.7,5.2)$) circle [radius=0.06];
\pause 	
\node at ($(current page.south west)+(8.5,4)$) [anchor=north west] {\includegraphics[scale=0.5]{graphics/who.png}};
\end{tikzpicture} 	
\end{frame}

\begin{frame}
\frametitle{Orbits - What's the Big Deal? (III)}
\begin{tikzpicture}[remember picture, overlay]
\draw [->,thick] (current page.south west) +(6,0.3) -- +(6,8);
\draw [->,thick] (current page.south west) +(0.3,4) -- +(12,4);	
\node at ($(current page.south west)+(2,7.5)$) {If $|\lambda_{\max}|<1$};   
\pause
\node at ($(current page.south west)+(2.8,1.5)$) {$\colInit{\vec{x}}$};
\draw[fill] ($(current page.south west)+(3,1.5)$) circle [radius=0.06];
\pause
\node at ($(current page.south west)+(6.8,3)$) {$\colTarget{\vec{y}}$};
\draw[fill] ($(current page.south west)+(7,3)$) circle [radius=0.06];
\pause
\draw[fill] ($(current page.south west)+(3.5,3)$) circle [radius=0.06];
\pause
\draw[fill] ($(current page.south west)+(4.5,4)$) circle [radius=0.06];
\pause
\draw[fill] ($(current page.south west)+(5.7,4.6)$) circle [radius=0.06];
\pause
\draw[fill] ($(current page.south west)+(7,4.2)$) circle [radius=0.06];
\pause
\draw[fill] ($(current page.south west)+(6.6,3.4)$) circle [radius=0.06];
\pause
\draw[fill] ($(current page.south west)+(5.7,3.1)$) circle [radius=0.06];
\pause
\draw[fill] ($(current page.south west)+(5.5,3.6)$) circle [radius=0.06];
\pause
\draw[fill] ($(current page.south west)+(5.8,4.1)$) circle [radius=0.06];
\pause 	
\node at ($(current page.south west)+(8.5,4)$) [anchor=north west] {\includegraphics[scale=0.5]{graphics/who.png}};
\end{tikzpicture} 	
\end{frame}

\begin{frame}
\frametitle{Orbits - What's the Big Deal? (IV)}
\begin{tikzpicture}[remember picture, overlay]
\node at ($(current page.south west)+(2,7.5)$) {If$^*$ $|\lambda_{\max}|=1$};   
\node at (current page.center) {\animategraphics[autoplay,loop,scale=0.5]{25}{Animation/test_(}{1000}{1399}};
\node at ($(current page.south west)+(1,1)$) [anchor=west] {\small * Assuming dominant eigenvalues are not roots of unity};   
\end{tikzpicture} 	   
\end{frame}



\begin{frame}
\frametitle{Talk Outline}
\begin{itemize}
\item \emc{gray}{Non-Boolean Reasoning:}
\begin{itemize} 
    \item \emc{gray}{Reasoning about Quality - quantitative specifications.}
	\item \emc{gray}{Regular Sensing - a new complexity measure for $\omega$-regular languages.}
\end{itemize}
\item Linear Dynamical Systems: 
\begin{itemize}
\item[\itemnow] \textbf{\emc{red}{The Polytope Collision Problem - input quantification in infinite-state systems.}}
\item[\itemlater] Semialgebraic invariant synthesis - certifiable algorithms for orbit problems.
\end{itemize}
\end{itemize}
\end{frame}

\begin{frame}
	\frametitle{The Polytope-Collision Problem}
	\begin{tikzpicture}[remember picture, overlay]
	\pause
	\node at ($(current page.south west)+(1,7)$) [anchor=north west] {{\begin{varwidth}{\linewidth}\begin{itemize}
			\setlength\itemsep{1.3em}
			\item<2-> We work in $\RR^d$.
			\item<3-> Initial polytope $\colInit{\P}$.
			\item<5-> Target polytope $\colTarget{\Q}$.		
			\item<6-> Matrix $A\in \RR^{d\times d}$
			\item<7-> Will the {\em Orbit} of $\colInit{\P}$ hit $\colTarget{\Q}$?
			\item<8-> i.e. Does there exist $n\in \NN$ such that $A^n\colInit{\P}\cap \colTarget{\Q}\neq \emptyset$?
			\end{itemize}\end{varwidth}}
	};
	
	\onslide<2->
	\draw [->,ultra thick] (current page.south west) +(9,3) -- +(9,7.5);
	\draw [->,ultra thick] (current page.south west) +(6,3.5) -- +(11.5,3.5);
	\onslide<3>{
		\node at ($(current page.south west)+(7,4)$) {\includegraphics[scale=0.1]{graphics/carbluet.png}};
		\node at ($(current page.south west)+(6.9,4)$) {$\colInit{\P}$};
	}
	\onslide<4->
	\node at ($(current page.south west)+(7,4)$) {\includegraphics[scale=0.1]{graphics/polycarT1.png}};
	\node at ($(current page.south west)+(6.9,4)$) {$\colInit{\P}$};
	\onslide<5->
	\node at ($(current page.south west)+(11,6.5)$) {\includegraphics[scale=0.25]{graphics/treeT.png}};
	\node at ($(current page.south west)+(11,5.15)$) {$\colTarget{\Q}$};
	\onslide<9->
	\node at ($(current page.south west)+(7,6)$) {\includegraphics[scale=0.1]{graphics/polycarT2.png}};
	\node at ($(current page.south west)+(6.9,6)$) {$A\colInit{\P}$};
	\onslide<10->
	\node at ($(current page.south west)+(9.5,3.5)$) {\includegraphics[scale=0.1]{graphics/polycarT3.png}};
	\node at ($(current page.south west)+(9.5,3.5)$) {$A^2\colInit{\P}$};   
	\onslide<11->
	\node at ($(current page.south west)+(10.1,6.2)$) {\includegraphics[scale=0.1]{graphics/polycarT4.png}};
	\node at ($(current page.south west)+(10.1,6)$) {$A^3\colInit{\P}$};   
	\end{tikzpicture} 	
\end{frame}



\begin{frame}
	\frametitle{Problem and Results}
	\pause
	\onslide<+->
	\begin{beamerboxesrounded}[upper=uppercolblue,lower=lowercolblue,shadow=true]{The Polytope-Collision Problem}
		Given polytopes $\colInit{\P}, \colTarget{\Q}\subseteq \RR^d$ and a matrix $A\in \QQ^{d\times d}$, decide whether there exists $n\in \NN$ such that $A^n\colInit{\P}\cap \colTarget{\Q}\neq \emptyset$.
	\end{beamerboxesrounded}
	\onslide<+->
	Results 	[S. Almagor, J. Ouaknine, and J. Worrell, \textbf{ICALP 2017}]:
	\onslide<+->	
	\begin{beamerboxesrounded}[upper=uppercolgreen,lower=lowercolgreen,shadow=true]{$\Large \emc{black}{\smiley}$ }
		For $d\le 3$, {\bf decidable} in PSPACE.
	\end{beamerboxesrounded}	
	\onslide<+->
	\begin{beamerboxesrounded}[upper=uppercolred,lower=lowercolred,shadow=true]{\Large $\emc{black}{\frownie}$}
		For $d\ge 4$, {\bf hard} with respect to long-standing number-theoretic open problems.
	\end{beamerboxesrounded}

\end{frame}

%\begin{frame}
%\frametitle{3D Polytope Collision Problem is in PSPACE\\ Proof Outline}
%\pause
%\begin{itemize}[<+->]
%\item \alt<2-6,9->{Reduce to the invertible-matrix case.}{\bf \emc{darkgreen}{Reduce to the invertible-matrix case.}}
%\item \alt<3-8>{Reason about intersecting polytopes.}{\bf \emc{darkgreen}{Reason about intersecting polytopes.}}
%\item Formulate the problem as a (symbolic) first-order sentence in the theory of the reals.
%\item Eliminate quantifiers to get systems of inequalities.
%\item Solve the systems.
%\end{itemize}
%\onslide<8>{
%	\begin{tikzpicture}[remember picture, overlay]
%	\node at ($(current page.center) + (0,1.5)$) [anchor=north] {\includegraphics[scale=0.5]{graphics/rabbit.jpg}};
%	\end{tikzpicture}
%}
% \end{frame}

% \begin{frame}
% \frametitle{Intersection of polytopes}
% \begin{itemize}
% \item<1-> Consider {\em tetrahedra} $\colInit{\T_1}$ and $\colTarget{\T_2}$.
% \item<2-> Observe that $\colInit{\T_1}$ intersects $\colTarget{\T_2}$ iff either of the following holds:
% \begin{itemize}
% \item<3-> An edge of $\colInit{\T_1}$ intersects a face of $\colTarget{\T_2}$\onslide<4->{, or \emc{red}{vice versa}.}
% \item<5-> A vertex of $\colInit{\T_1}$ lies in $\colTarget{\T_2}$\onslide<6->{, or \emc{red}{vice versa}.}
% \end{itemize}
% \item<7-> We show that it is enough to solve the following:
%
% \begin{beamerboxesrounded}[upper=uppercolblue,lower=lowercolblue,shadow=true]{3DPCP:}
% 		Given polytopes $\colInit{\P}, \colTarget{\Q}$ in $\RR^3$ and a matrix $A\in \QQ^{3\times 3}$, decide whether there exists $n\in \NN$ such that an \textbf{edge} of $A^n\colInit{\P}$ intersects a \textbf{face} of $\colTarget{\Q}$.
% 	\end{beamerboxesrounded}
% \end{itemize}
%\begin{center}
%\onslide<3->{\includegraphics[scale=0.22]{graphics/tetra1.png}}
%\onslide<4->{\includegraphics[scale=0.22]{graphics/tetra2.png}}
%\onslide<5->{\includegraphics[scale=0.18]{graphics/tetra3.png}}
%\onslide<6->{\includegraphics[scale=0.18]{graphics/tetra4.png}}
%\end{center}
% \end{frame}
 
%\begin{frame}
% \frametitle{3DPCP is in PSPACE - Proof Outline}
% \begin{itemize}
% \item Reduce to the invertible-matrix case.
% \item Reason about intersecting polytopes.
% \item {\bf \emc{darkgreen}{Formulate the problem as a (symbolic) first-order sentence in the theory of the reals.}}
% \item Eliminate quantifiers to get systems of inequalities.
% \item Solve the systems.
% \end{itemize}
%\end{frame}

\begin{frame}
 \frametitle{3D Polytope Collision Problem - Simplification}
For the purpose of this talk, we \emc{orange}{assume} the following:

\onslide<2->
$\emc{orange}{\bullet}$ $\colInit{\P}$ is a \emph{segment}: $\colInit{{\cal P}}= \set{\colInit{\vec{u}}+\lambda \colInit{\vec{v}}: \lambda\in [0,1]}$.

\onslide<3->
Recall that $A\in \QQ^{3x3}$.

\onslide<4->
$\emc{orange}{\bullet}$ Assume $A$ is diagonalizable. Then, we can write
$$A=B\begin{pmatrix}
\alpha & 0 & 0\\
0 & \alt<1-5>{\beta}{\conj{\alpha}} & 0\\
0 &0 & \alt<1-7>{\rho}{|\alpha|}\\
\end{pmatrix}B^{-1}$$
\onslide<5->
$\emc{orange}{\bullet}$ Assume that $\alpha$ is complex, so $\beta=\conj{\alpha}$ \onslide<7->{ and $\rho$ is real with $\rho=|\alpha|$.}

\onslide<9->

$\emc{orange}{\bullet}$ Assume $\gamma=\frac{\alpha}{|\alpha|}$ is \textbf{not} a root of unity.

\onslide<10->

All eigenvalues are {\bf algebraic} numbers.
\onslide<11->

For every vector $\vec{x}$, we have that 
$$(A^n\vec{x})_i=a_i \alpha^n+\conj{a_i} \bar{\alpha}^n+b_i |\alpha|^n$$
\end{frame}
 
 \newcommand{\foR}{{\text{FO}[\RR]}}
 
\begin{frame}
  \frametitle{Formulating the Problem in FO}
 	
 	Since $\colInit{{\cal P}}= \set{\colInit{\vec{u}}+\lambda \colInit{\vec{v}}: \lambda\in [0,1]}$, our problem can be written as:
 	\onslide<2->
 	\begin{beamerboxesrounded}[upper=uppercolblue,lower=lowercolblue,shadow=true]{3DPCP - simplification rewrite:}
 			Does there exist $n\in \NN$ such that the following sentence is true:
 			$\exists \lambda : $ $0\le \lambda\le 1\ \wedge A^n \colInit{\vec{u}}+\lambda A^n\colInit{\vec{v}}\in \colTarget{{\cal Q}}$
 		\end{beamerboxesrounded}
	\onslide<3->
	If $n$ is fixed, this is$^{*}$ a sentence in the \emph{first order theory of the reals}!
	\onslide<4->
	\begin{center}
	 	\includegraphics[scale=0.3]{graphics/detour.jpg}
	\end{center}

  \end{frame}
  
\begin{frame}
	\frametitle{First Order Theory of the Reals -- Crash Course}
	\begin{itemize}[<+->]
		\item A sentence in $\foR$ consists of:
		\begin{itemize}
			\item Real-valued variables, possible quantified.
			\item Boolean operators (i.e., $\wedge,\vee,\neg$).
			\item Predicates of the form $p(x_1,\ldots,x_n)>0$ where $p$ is a polynomial with \emph{rational} coefficients.
		\end{itemize}
	\item Example: $\exists x : x^2+ 2x-1 >0 \wedge x>0$
	\item Admits \textbf{Quantifier Elimination}:
	
	\begin{beamerboxesrounded}[upper=uppercollem,lower=lowercollem,shadow=true]{$\foR$ Quantifier Elimination [Tarski-–Seidenberg]}
		Every $\foR$ sentence can be transformed to an equivalent quantifier-free sentence.
	\end{beamerboxesrounded}

	\item Examples:
	\begin{itemize}
		\item The sentence above is equivalent to \emph{True}.
		\item $\exists x: ax^2+bx+c=0$ is equivalent to $b^2-4ac\ge 0$.
	\end{itemize}
	\end{itemize}
\end{frame}
  
  \begin{frame}
  	\frametitle{Formulating the Problem in FO}
  	
  	Since $\colInit{{\cal P}}= \set{\colInit{\vec{u}}+\lambda \colInit{\vec{v}}: \lambda\in [0,1]}$, our problem can be written as:
  	\begin{beamerboxesrounded}[upper=uppercolblue,lower=lowercolblue,shadow=true]{3DPCP - simplification rewrite:}
  		Does there exist $n\in \NN$ such that the following sentence is true:
  		\alt<1-6>{$\exists \lambda : $ $0\le \lambda\le 1\ \wedge A^n \colInit{\vec{u}}+\lambda A^n\colInit{\vec{v}}\in \colTarget{{\cal Q}}$}{$\exists \emc{red}{\lambda} : $ $0\le \emc{red}{\lambda}\le 1\ \wedge A^n \colInit{\vec{u}}+\emc{red}{\lambda} A^n\colInit{\vec{v}}\in \colTarget{{\cal Q}}$}
  	\end{beamerboxesrounded}
  	If $n$ is fixed, this is$^{*}$ a sentence in $\foR$!
  	\only<1>{
  		\begin{center}
  		\includegraphics[scale=0.7]{graphics/enddetour.jpg}
  	\end{center}}
  
	  \onslide<2->
	  Recall: every coordinate of $A^n \colInit{\vec{u}}$ (or $A^n\colInit{\vec{v}}$) is of the form:
	  $$(A^n\vec{u})_i=a_i \alpha^n+\conj{a_i} \bar{\alpha}^n+b_i |\alpha|^n$$
	  
	\begin{itemize}
	  \item<3-> Admits quantifier elimination.
	  \item<4-> However, we need to eliminate quantifiers \emc{darkgreen}{symbolically},\\ since $n$ is a parameter.
	  \item<5-> The CAD algorithm is very complicated to analyze.
	  \item<6-> Notice that the expressions above are \emc{red}{linear}!
	  \item<8-> We can symbolically apply Fourier-Motzkin elimination.
	\end{itemize}		
  \end{frame}
	
 \begin{frame}
	\frametitle{Fourier-Motzkin Elimination Result (Partial)}
	\begin{tikzpicture}[remember picture, overlay]
	\node at ($(current page.center)+(0,-0.5)$)[anchor=center] {
		\includegraphics[scale=0.6]{graphics/FMResult.png}
	};
	\end{tikzpicture}
\end{frame}

 \begin{frame}
 \frametitle{Fourier-Motzkin Elimination Result}
	Recall: every coordinate of $A^n \colInit{\vec{u}}$ (or $A^n\colInit{\vec{v}}$) is of the form:
	\vspace*{-8pt}
	$$(A^n\vec{u})_i=a_i \alpha^n+\conj{a_i} \bar{\alpha}^n+b_i |\alpha|^n$$
	\vspace*{-15pt}

	\onslide<2->
	Eliminating $\lambda$ from $A^n\colInit{\vec{u}}+\lambda A^n\colInit{\vec{v}}$ results (after normalization) in a quadratic expression:
	\onslide<3->
	
	\begin{beamerboxesrounded}[upper=uppercolblue,lower=lowercolblue,shadow=true]{Theorem:}
	The sentence is equivalent to finding the solution (i.e., $n\in\NN$) to a conjunction of expressions of the form
	\vspace*{-8pt}
	\begin{align*}
		& B \gamma^{2n}+ \conj{B}\conj{\gamma}^{2n}+C\gamma^n+\conj{C}\conj{\gamma}^n+D+r(n)>0 \\
	\end{align*}
	
	\vspace*{-25pt}
	with $r(n)\to 0$ exponentially fast, and $\gamma=\frac{\alpha}{|\alpha|}$ is not a root of unity.
 	\end{beamerboxesrounded}
 	
 	\onslide<4>{
 	 		\begin{tikzpicture}[remember picture, overlay]
 	 			\node at ($(current page.center)+(0,-0.7)$)[anchor=center] {\includegraphics[scale=0.5]{graphics/believe.jpg}};
 	 			\end{tikzpicture}
 	 	}
 	 \onslide<5-> 
 	  	Solving this is our main technical challenge \\
 	  	(i.e., finding $n\in \NN$ that satisfies all the expressions).
 \end{frame}

%\begin{frame}
%	\frametitle{Solution Sketch}
%	\begin{beamerboxesrounded}[upper=uppercolblue,lower=lowercolblue,shadow=true]{A conjunction of expressions of the form:}
%	\vspace*{-15pt}
%	\begin{align*}
%		& B \gamma^{2n}+ \conj{B}\conj{\gamma}^{2n}+C\gamma^n+\conj{C}\conj{\gamma}^n+D+r(n)>0
%	\end{align*}
%	\vspace*{-20pt}
%	
%		with $r(n)\to 0$ exponentially fast, and $\gamma=\frac{\alpha}{|\alpha|}$ is not a root of unity.
%	\end{beamerboxesrounded}
%	\onslide<+->
%	\begin{beamerboxesrounded}[upper=uppercolgreen,lower=lowercolgreen,shadow=true]{Main Lemma (Intuitive Version):}
%		We can drop $r(n)$ in the above, after some $N\in \NN$.
%	\end{beamerboxesrounded}
%\end{frame}


\begin{frame}
 \frametitle{Solution Sketch}
	 	\onslide<+->
		\begin{beamerboxesrounded}[upper=uppercolblue,lower=lowercolblue,shadow=true]{}
			\vspace*{-15pt}
			\begin{align*}
			& B \emc{red}{\gamma^{2n}}+ \conj{B}\emc{red}{\conj{\gamma}^{2n}}+C\emc{red}{\gamma^n}+\conj{C}\emc{red}{\conj{\gamma}^n}+D+r(n)>0
			\end{align*}
			
			\vspace*{-20pt}
			with $r(n)\to 0$ exponentially fast, and $\emc{red}{\gamma}=\frac{\alpha}{|\alpha|}$ is not a root of unity.
		\end{beamerboxesrounded}
		\onslide<+->
		\begin{beamerboxesrounded}[upper=uppercolgreen,lower=lowercolgreen,shadow=true]{Main Lemma (Intuitive Version):}
			We can drop $r(n)$ in the above, after some $N\in \NN$.
		\end{beamerboxesrounded}
	 	\onslide<+->
	 	$\set{\emc{red}{\gamma^n}:n\in \NN}$ is dense in the unit circle $\set{z\in \CC: |z|=1}$.\\
	 	\onslide<+->
	 	Consider $f:\CC\to \RR$ defined by $f(\emc{blue}{z})=B \emc{blue}{z^2}+ \conj{B}\emc{blue}{\conj{z}^{2}}+C\emc{blue}{z}+\conj{C}\emc{blue}{\conj{z}}+D$.\\
	 	\onslide<+->
		$f$ has \emc{blue}{$4$ roots} on the unit circle. How close does $\emc{red}{\gamma^n}$ get to them?
		\begin{center}
		\begin{tikzpicture}
			 	%\draw [->,thick] (current page.south west) +(6,0.3) -- +(6,8);
			 	%\draw [->,thick] (current page.south west) +(0.3,4) -- +(12,4);	
		 		%\fill[red!20!white] ($(current page.center)+(-6.6,1.52)$) rectangle ($(current page.center)+(4.65,-2.2)$);
		 		\draw[gray] ($(current page.center)+(-1.2,0)$) -- ($(current page.center)+(1.2,0)$);
		 		\draw[gray] ($(current page.center)+(0,1.2)$) -- ($(current page.center)+(0,-1.2)$);
		 		\draw ($(current page.center)$) circle [radius=1.2];
		 		%\draw[step=1,black,very thin] ($(current page.center)+(-8,-6)$) grid ($(current page.center)+(4,4)$);
				\draw[fill,blue] ($(current page.center)+(60:1.2)$) circle [radius=0.1];
				\draw[fill,blue] ($(current page.center)+(130:1.2)$) circle [radius=0.1];
				\draw[fill,blue] ($(current page.center)+(210:1.2)$) circle [radius=0.1];
				\draw[fill,blue] ($(current page.center)+(250:1.2)$) circle [radius=0.1];
		 		\foreach \a in {1,...,7}
			 		{
			 		\pause
			 		\draw[fill,red] ($(current page.center)+(75*\a:1.2)$) circle [radius=0.1];
			 		}
		\end{tikzpicture} 		
		\end{center}
			 	
		
 \end{frame}

\begin{frame}
 \frametitle{Separating Powers}
	 	\onslide<+->
	 	\begin{beamerboxesrounded}[upper=uppercolred,lower=lowercolred,shadow=true]
	 	%{Lemma [J.Ouaknine and J.Worrell, ICALP 2014]}
	 	{Lemma [Consequence of the Baker-W\"ustholz Theorem]:}
	 		There exists $D\in \NN$ such that for every $\emc{red}{\zeta},\emc{blue}{\xi}\in \AA$ of modulus 1, if $\emc{red}{\zeta}^n\neq \emc{blue}{\xi}$, then $|\emc{red}{\zeta}^n-\emc{blue}{\xi}|>n^{-(\norm{\emc{red}{\zeta}}+\norm{\emc{blue}{\xi}})^D}$.	 							
	 	\end{beamerboxesrounded}
	 	% % % % % % % % % % % %
		\onslide<+->
	 	\begin{tikzpicture}
	 	%\draw [->,thick] (current page.south west) +(6,0.3) -- +(6,8);
	 	%\draw [->,thick] (current page.south west) +(0.3,4) -- +(12,4);	
 		\fill[green!20!white] ($(current page.center)+(-6.6,1.52)$) rectangle ($(current page.center)+(4.3,-2.2)$);
 		\draw[gray] ($(current page.center)+(-3.5,-0.5)$) -- ($(current page.center)+(0.5,-0.5)$);
 		\draw[gray] ($(current page.center)+(-1.5,-2.2)$) -- ($(current page.center)+(-1.5,1.5)$);
 		\draw ($(current page.center)+(-1.5,-0.5)$) circle [radius=1.5];
 		%\draw[step=1,black,very thin] ($(current page.center)+(-8,-6)$) grid ($(current page.center)+(4,4)$);
		\draw[fill,blue] ($(current page.center)+(-1.5,-0.5)+(60:1.5)$) circle [radius=0.1];
 		\node at ($(current page.center)+(-1.5,-0.5)+(60:1.75)$) {$\emc{blue}{\xi}$};
 		\onslide<+->
 		\draw[fill,red] ($(current page.center)+(-1.5,-0.5)+(15:1.5)$) circle [radius=0.1];
 		\node at ($(current page.center)+(-1.5,-0.5)+(15:1.75)$) {$\emc{red}{\zeta}$};
 		\draw[orange] ($(current page.center)+(-1.5,-0.5)+(15:1.5)$) -- ($(current page.center)+(-1.5,-0.5)+(60:1.5)$);
 		\onslide<+->
		\draw[fill,red] ($(current page.center)+(-1.5,-0.5)+(90:1.5)$) circle [radius=0.1];
 		\node at ($(current page.center)+(-1.5,-0.5)+(90:1.75)$) {$\emc{red}{\zeta^2}$};
 		\draw[orange] ($(current page.center)+(-1.5,-0.5)+(90:1.5)$) -- ($(current page.center)+(-1.5,-0.5)+(60:1.5)$);
 		\onslide<+->
		\draw[fill,red] ($(current page.center)+(-1.5,-0.5)+(165:1.5)$) circle [radius=0.1];
 		\node at ($(current page.center)+(-1.5,-0.5)+(165:1.75)$) {$\emc{red}{\zeta^3}$}; 	
 		\draw[orange] ($(current page.center)+(-1.5,-0.5)+(165:1.5)$) -- ($(current page.center)+(-1.5,-0.5)+(60:1.5)$);
 		\onslide<+->
		\draw[fill,red] ($(current page.center)+(-1.5,-0.5)+(45:1.5)$) circle [radius=0.1];
 		\node at ($(current page.center)+(-1.5,-0.5)+(45:1.75)$) {$\emc{red}{\zeta^n}$};
 		\draw[orange] ($(current page.center)+(-1.5,-0.5)+(45:1.5)$) -- ($(current page.center)+(-1.5,-0.5)+(60:1.5)$);
 		%\draw[fill] ($(current page.center)$) circle [radius=0.3];
	 	\end{tikzpicture} 		
	 	
	 	% % % % % % % % % % % % 
 		\onslide<7->{
		Also starring: symbolic Taylor approximations, decidability of $\foR$, root-separation bounds, and PosSLP.}
 
 \end{frame} %3DPCP

\begin{frame}
\frametitle{Talk Outline}
\begin{itemize}
	\item \emc{gray}{Non-Boolean Reasoning:}
	\begin{itemize} 
		\item \emc{gray}{Reasoning about Quality - quantitative specifications.}
		\item \emc{gray}{Regular Sensing - a new complexity measure for $\omega$-regular languages.}
	\end{itemize}
	\item Linear Dynamical Systems: 
	\begin{itemize}
	\item \emc{gray}{The Polytope Collision Problem - input quantification in infinite-state systems.}
	\item[\itemnow] \textbf{\emc{red}{Semialgebraic invariant synthesis - certifiable algorithms for orbit problems.}}
	\end{itemize}
\end{itemize}
\end{frame}

\begin{frame}
	\frametitle{Semialgebraic Invariants for Halfspace Reachability}
	Recall:
	\begin{beamerboxesrounded}[upper=lowercoldarkred,lower=uppercoldarkred,shadow=true]
		{Given a vector $\colInit{\vec{x}}\in \QQ^d$, a halfspace $\colTarget{V}$ and a matrix $A\in \QQ^{d\times d}$, decide whether there exists $n\in \NN$ such that $A^n\colInit{x}\in \colTarget{V}$.}
			Decidability status is open. Hard in general. 
	\end{beamerboxesrounded}
	\vspace*{-18pt}
	\onslide<+->
	\begin{itemize}
		\item<+-> For ``yes'' instances - simple certificate ($n\in \NN$).
		\item<+-> Can we say something about ``no'' instances? 
		\item<+-> Idea: find a set $\Inv$ such that:
		\begin{itemize}
			\item $\Inv$ contains \alt<1-5>{the orbit $\set{A^n \colInit{\vec{x}}:n\in \NN}$}{$\colInit{\vec{x}}$}.
			\item $\Inv$ is disjoint from the halfspace $\colTarget{V}$.
			\item $\Inv$ is ``simple'' to describe.
			\item<+-> $\Inv$ is invariant under $A$: $A\Inv\subseteq \Inv$.
		\end{itemize} 
	\end{itemize}
	\onslide<+->	
	\onslide<+->
	\begin{beamerboxesrounded}[upper=uppercolgreen,lower=lowercolgreen,shadow=true]{Theorem [S. Almagor, D. Chistikov, J. Ouaknine, J. Worrell]:}
		{Given a vector $\colInit{\vec{x}}$, a halfspace $\colTarget{V}$ and a diagonalizable matrix $A$, we can compute $\Inv$, if it exists.}
	\end{beamerboxesrounded}	
\end{frame}

\begin{frame}
\frametitle{Semialgebraic Invariants - Proof Sketch (1)}

\begin{tikzpicture}[overlay]
	\only<1->{\node at ($(current page.south west)+(-9.6,2)+(10,1)$) [anchor=north west]{Consider the orbit of $\colInit{\vec{x}}$, we can extract a ``cone'' out of it:};}
	\only<2>{\node at ($(current page.south west)+(-8.8,1.5)+(10,1)$) [anchor=north west]{\includegraphics[scale=0.3]{Cone/Points.png}};}
	\only<3>{\node at ($(current page.south west)+(-8,1.7)+(10,1)$) [anchor=north west]{\includegraphics[scale=0.3]{Cone/Lines.png}};}
	\only<4>{\node at ($(current page.south west)+(-8.6,1.3)+(10,1)$) [anchor=north west]{\includegraphics[scale=0.3]{Cone/Cone.png}};}
\end{tikzpicture} 


\end{frame}

\begin{frame}
\frametitle{Semialgebraic Invariants - Proof Sketch (2)}
\begin{itemize}[<+->]
\item We prove that every semialgebraic invariant must contain the entire cone from some ``height''.
\item Thus, it is enough to check if the cone intersects the halfspace infinitely often.
\item If the cone is semialgebraic - this can be done using decidability of the FO theory of the reals.
\end{itemize}
\includegraphics[scale=0.15]{Cone/Cone.png}
\begin{itemize}[<+->]
\item Problem: the cone might not be semialgebraic.
\item Solution: work with a ``fat'' cone that is semialgebraic.
\end{itemize}
\end{frame} %semialgebraic inv.


\section{Summary}
\stepcounter{subsection}

\begin{frame}
 \frametitle{Research Outlook}
{\small
\begin{itemize}[<+->]
\item Formal methods need to be extended beyond the Boolean framework and beyond finite-state systems.
\item We need to convince designers of correctness and incorrectness.
\item Reasoning about infinite-state systems/quantitative specifications requires some structure (algebraic or other).
\item Rich domains bring a variety of (cool) tools to formal methods.
\item In a rich domain, it is (potentially) easier to provide certificates.
\item We get a whole new set of challenges that are not combinatorial (unlike the traditional setting).
\item Giving back to the community: formal methods may shed some light on open problems in other fields.
\end{itemize}
}
\end{frame}


 \begin{frame}
 \frametitle{}
 \begin{center}
 {\Huge Thank You!}
 
 \includegraphics[scale=0.5]{graphics/accident.png}
 \end{center}
 \end{frame}
  
 \end{document}
 

 
