\begin{frame}
\frametitle{Talk Outline}
\begin{itemize}
\item \emc{gray}{Non-Boolean Reasoning:}
\begin{itemize} 
    \item \emc{gray}{Reasoning about Quality - quantitative specifications.}
	\item \emc{gray}{Regular Sensing - a new complexity measure for $\omega$-regular languages.}
\end{itemize}
\item[\itemnow] \textbf{\emc{red}{Linear Dynamical Systems:}}
\begin{itemize}
\item[\itemlater] The Polytope Collision Problem - input quantification in infinite-state systems.
\item[\itemlater] Semialgebraic invariant synthesis - certifiable algorithms for orbit problems.
\end{itemize}
\end{itemize}
\end{frame}


\begin{frame}
	\frametitle{Linear Dynamical Systems}
	\begin{itemize}
		\item We consider a Linear Dynamical System:
		\onslide<+->
		\begin{itemize}[<+->]
			\item Start with a vector $\colInit{\vec{x}}\in \RR^d$.
			\item Progress by applying a matrix $A\in \RR^{d\times d}$.
			\item Reason about the {\em Orbit} ${\cal O}=\set{A^n \colInit{\vec{x}}:n\in \NN}$.
		\end{itemize}
	\item<+-> Models an infinite state machine.\\
	\hspace*{2.5cm}
		\begin{beamerboxesrounded}[upper=lowercolblue,lower=lowercolblue,shadow=true, width=3cm]{}
			\texttt{init} $\colInit{\vec{x}}$\\
			\texttt{while}({\em condition})\\
			\hspace*{0.5cm} $\colInit{\vec{x}}\leftarrow A\colInit{\vec{x}}$
		\end{beamerboxesrounded}
	\onslide<+->
	\item Properties to consider:
%	{\footnotesize 
%			\onslide<+->
%		\item[-] Vector-Reachability: Given a vector $\colTarget{\vec{y}}\in \RR^d$, $\exists n\in \NN$ s.t. $A^n \colInit{\vec{x}}=\colTarget{\vec{y}}$?
%			\onslide<+->
%		\item[-] Set-Reachability: Given a target set $\colTarget{T}\subseteq \RR^d$, $\exists n\in \NN$ s.t. $A^n \colInit{\vec{x}}\in \colTarget{T}$?
%			\onslide<+->
%		\item[-] (Non)Safety: Given sets $\colInit{S},\colTarget{T}\subseteq \RR^d$, $\exists \colInit{\vec{x}}\in \colInit{S}, \exists n\in \NN$ s.t.  $A^n \colInit{\vec{x}}\in \colTarget{T}$?
%			\onslide<+->
%		\item[-] Termination: Given sets $\colInit{S},\colTarget{T}\subseteq \RR^d$, $\forall \colInit{\vec{x}}\in \colInit{S}, \exists n\in \NN$ s.t.  $A^n \colInit{\vec{x}}\in \colTarget{T}$?
%	}
	{\footnotesize 
		\setlength{\tabcolsep}{1pt}
		\begin{tabular}{l l l l}
		\onslide<+->
		Vector-Reachability: & Given a vector $\colTarget{\vec{y}}\in \RR^d$, & &  $\exists n\in \NN$ s.t. $A^n \colInit{\vec{x}}=\colTarget{\vec{y}}$? \\ 
		\onslide<+->
		 Set-Reachability: & Given a set $\colTarget{T}\subseteq \RR^d$, & & $\exists n\in \NN$ s.t. $A^n \colInit{\vec{x}}\in \colTarget{T}$? \\ 
		\onslide<+->		 
		 (Non)Safety: & Given sets $\colInit{S},\colTarget{T}\subseteq \RR^d$, &  $\exists \colInit{\vec{x}}\in \colInit{S}$, & $\exists n\in \NN$ s.t.  $A^n \colInit{\vec{x}}\in \colTarget{T}$?\\
		\onslide<+->		 
		 Termination: & Given sets $\colInit{S},\colTarget{T}\subseteq \RR^d$, & $\forall \colInit{\vec{x}}\in \colInit{S},$ & $\exists n\in \NN$ s.t.  $A^n \colInit{\vec{x}}\in \colTarget{T}$?
		\end{tabular} 
	}	
	\end{itemize}
\end{frame}

\begin{frame}
	\frametitle{Orbit Problems - History}
	\pause
	\onslide<2->{
		\begin{beamerboxesrounded}[upper=lowercolblue,lower=lowercolgreen,shadow=true]
			{Given vectors $\colInit{\vec{x}}, \colTarget{\vec{y}}\in \QQ^d$ and a matrix $A\in \QQ^{d\times d}$, decide whether there exists $n\in \NN$ such that $A^n\colInit{\vec{x}}=\colTarget{\vec{y}}$.}
			\onslide<3->{ 
				Shown to be decidable in {\bf P} by Kannan and Lipton in [STOC 1980, JACM 1986].
			}
		\end{beamerboxesrounded}
	}
	\onslide<4->{
		\begin{beamerboxesrounded}[upper=lowercolblue,lower=lowercolgreen,shadow=true]
			{Given a vector $\colInit{\vec{x}}\in \QQ^d$, a halfspace $\colTarget{V}$ and a matrix $A\in \QQ^{d\times d}$, decide whether there exists $n\in \NN$ such that $A^n\colInit{\vec{x}}\in \colTarget{V}$.}
			\onslide<7->{
				Hard-open$^*$ in general. 				
				Solvable when $\colTarget{V}$ has a low dimension. 
				
				[V. Chonev, J. Ouaknine, and J. Worrell. STOC 2013, J.ACM 2016]
			}
		\end{beamerboxesrounded}
	}
	\onslide<8->{
		\begin{beamerboxesrounded}[upper=lowercolblue,lower=lowercolgreen,shadow=true]
			{Given a vector $\colInit{\vec{x}}\in \QQ^d$, a polytope $\colTarget{\Q}$ and a matrix $A\in \QQ^{d\times d}$, decide whether there exists $n\in \NN$ such that $A^n\colInit{\vec{x}}\in \colTarget{\Q}$.}
			\onslide<9->{
				Hard-open$^*$ in general. 				
				Solvable when $\colTarget{\Q}$ has dimension $\le 3$. 
				
				[V. Chonev, J. Ouaknine, and J. Worrell. SODA 2015]
			}
		\end{beamerboxesrounded}	
	}
	\only<5>{\begin{tikzpicture}[overlay]
	\node at ($(current page.center)-(1,1)$) [anchor=center, rotate=10] {\includegraphics[scale=0.4]{graphics/newspaper.pdf}}; 
	\end{tikzpicture} 	}	
	\onslide<7->{\small * w.r.t. long-standing number-theoretic open problems (Diophantine approximation)}	
	
\end{frame}   

\begin{frame}
\frametitle{Orbits - What's the Big Deal? (I)} 
\begin{itemize}[<+->]
	\item Consider a matrix $A$ and a vector $\colInit{\vec{x}}$.
	\item Assume $A$ is diagonalizable, then we can write 
	\onslide<+->
	$$A=P^{-1}\begin{pmatrix}
	\lambda_1 & & \\
	& \ddots &\\
	& & \lambda_d
	\end{pmatrix}P\ \text{ with } \lambda_i\in \CC.$$ 
	\item Thus, every {\em coordinate} of $A^n\colInit{\vec{x}}$ is of the form
	$$a_1\lambda^n_1+\ldots+a_d\lambda^n_d.$$
	\item This is {\em dominated} by the maximal modulus $|\lambda_{\max}|$.
	\item Smaller eigenvalues have an {\em exponentially decreasing} effect.
\end{itemize}
\end{frame}

\begin{frame}
\frametitle{Orbits - What's the Big Deal? (II)}
\begin{tikzpicture}[remember picture, overlay]
\draw [->,thick] (current page.south west) +(6,0.3) -- +(6,8);
\draw [->,thick] (current page.south west) +(0.3,4) -- +(12,4);	
\pause
\node at ($(current page.south west)+(2,7.5)$) {If $|\lambda_{\max}|>1$};   
\pause
\node at ($(current page.south west)+(4.8,3)$) {$\colInit{\vec{x}}$};
\draw[fill] ($(current page.south west)+(5,3)$) circle [radius=0.06];
\pause
\node at ($(current page.south west)+(8.8,5)$) {$\colTarget{\vec{y}}$};
\draw[fill] ($(current page.south west)+(9,5)$) circle [radius=0.06];
\pause
\draw[fill] ($(current page.south west)+(5.5,2.6)$) circle [radius=0.06];
\pause
\draw[fill] ($(current page.south west)+(6.3,2.3)$) circle [radius=0.06];
\pause
\draw[fill] ($(current page.south west)+(7,3)$) circle [radius=0.06];
\pause
\draw[fill] ($(current page.south west)+(6.8,4.2)$) circle [radius=0.06];
\pause
\draw[fill] ($(current page.south west)+(5.7,4.8)$) circle [radius=0.06];
\pause
\draw[fill] ($(current page.south west)+(4,4.1)$) circle [radius=0.06];
\pause
\draw[fill] ($(current page.south west)+(4.1,2)$) circle [radius=0.06];
\pause
\draw[fill] ($(current page.south west)+(6.5,1.5)$) circle [radius=0.06];
\pause
\draw[fill] ($(current page.south west)+(8.3,2.5)$) circle [radius=0.06];
\pause
\draw[fill] ($(current page.south west)+(9.7,5.2)$) circle [radius=0.06];
\pause 	
\node at ($(current page.south west)+(8.5,4)$) [anchor=north west] {\includegraphics[scale=0.5]{graphics/who.png}};
\end{tikzpicture} 	
\end{frame}

\begin{frame}
\frametitle{Orbits - What's the Big Deal? (III)}
\begin{tikzpicture}[remember picture, overlay]
\draw [->,thick] (current page.south west) +(6,0.3) -- +(6,8);
\draw [->,thick] (current page.south west) +(0.3,4) -- +(12,4);	
\node at ($(current page.south west)+(2,7.5)$) {If $|\lambda_{\max}|<1$};   
\pause
\node at ($(current page.south west)+(2.8,1.5)$) {$\colInit{\vec{x}}$};
\draw[fill] ($(current page.south west)+(3,1.5)$) circle [radius=0.06];
\pause
\node at ($(current page.south west)+(6.8,3)$) {$\colTarget{\vec{y}}$};
\draw[fill] ($(current page.south west)+(7,3)$) circle [radius=0.06];
\pause
\draw[fill] ($(current page.south west)+(3.5,3)$) circle [radius=0.06];
\pause
\draw[fill] ($(current page.south west)+(4.5,4)$) circle [radius=0.06];
\pause
\draw[fill] ($(current page.south west)+(5.7,4.6)$) circle [radius=0.06];
\pause
\draw[fill] ($(current page.south west)+(7,4.2)$) circle [radius=0.06];
\pause
\draw[fill] ($(current page.south west)+(6.6,3.4)$) circle [radius=0.06];
\pause
\draw[fill] ($(current page.south west)+(5.7,3.1)$) circle [radius=0.06];
\pause
\draw[fill] ($(current page.south west)+(5.5,3.6)$) circle [radius=0.06];
\pause
\draw[fill] ($(current page.south west)+(5.8,4.1)$) circle [radius=0.06];
\pause 	
\node at ($(current page.south west)+(8.5,4)$) [anchor=north west] {\includegraphics[scale=0.5]{graphics/who.png}};
\end{tikzpicture} 	
\end{frame}

\begin{frame}
\frametitle{Orbits - What's the Big Deal? (IV)}
\begin{tikzpicture}[remember picture, overlay]
\node at ($(current page.south west)+(2,7.5)$) {If$^*$ $|\lambda_{\max}|=1$};   
\node at (current page.center) {\animategraphics[autoplay,loop,scale=0.5]{25}{Animation/test_(}{1000}{1399}};
\node at ($(current page.south west)+(1,1)$) [anchor=west] {\small * Assuming dominant eigenvalues are not roots of unity};   
\end{tikzpicture} 	   
\end{frame}
