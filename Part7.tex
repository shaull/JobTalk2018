\begin{frame}
\frametitle{Talk Outline}
\begin{itemize}
	\item \emc{gray}{Non-Boolean Reasoning:}
	\begin{itemize} 
		\item \emc{gray}{Reasoning about Quality - quantitative specifications.}
		\item \emc{gray}{Regular Sensing - a new complexity measure for $\omega$-regular languages.}
	\end{itemize}
	\item Linear Dynamical Systems: 
	\begin{itemize}
	\item \emc{gray}{The Polytope Collision Problem - input quantification in infinite-state systems.}
	\item[\itemnow] \textbf{\emc{red}{Semialgebraic invariant synthesis - certifiable algorithms for orbit problems.}}
	\end{itemize}
\end{itemize}
\end{frame}

\begin{frame}
	\frametitle{Semialgebraic Invariants for Halfspace Reachability}
	Recall:
	\begin{beamerboxesrounded}[upper=lowercoldarkred,lower=uppercoldarkred,shadow=true]
		{Given a vector $\colInit{\vec{x}}\in \QQ^d$, a halfspace $\colTarget{V}$ and a matrix $A\in \QQ^{d\times d}$, decide whether there exists $n\in \NN$ such that $A^n\colInit{x}\in \colTarget{V}$.}
			Decidability status is open. Hard in general. 
	\end{beamerboxesrounded}
	\vspace*{-18pt}
	\onslide<+->
	\begin{itemize}
		\item<+-> For ``yes'' instances - simple certificate ($n\in \NN$).
		\item<+-> Can we say something about ``no'' instances? 
		\item<+-> Idea: find a set $\Inv$ such that:
		\begin{itemize}
			\item $\Inv$ contains \alt<1-5>{the orbit $\set{A^n \colInit{\vec{x}}:n\in \NN}$}{$\colInit{\vec{x}}$}.
			\item $\Inv$ is disjoint from the halfspace $\colTarget{V}$.
			\item $\Inv$ is ``simple'' to describe.
			\item<+-> $\Inv$ is invariant under $A$: $A\Inv\subseteq \Inv$.
		\end{itemize} 
	\end{itemize}
	\onslide<+->	
	\onslide<+->
	\begin{beamerboxesrounded}[upper=uppercolgreen,lower=lowercolgreen,shadow=true]{Theorem [S. Almagor, D. Chistikov, J. Ouaknine, J. Worrell]:}
		{Given a vector $\colInit{\vec{x}}$, a halfspace $\colTarget{V}$ and a diagonalizable matrix $A$, we can compute $\Inv$, if it exists.}
	\end{beamerboxesrounded}	
\end{frame}

\begin{frame}
\frametitle{Semialgebraic Invariants - Proof Sketch (1)}

\begin{tikzpicture}[overlay]
	\only<1->{\node at ($(current page.south west)+(-9.6,2)+(10,1)$) [anchor=north west]{Consider the orbit of $\colInit{\vec{x}}$, we can extract a ``cone'' out of it:};}
	\only<2>{\node at ($(current page.south west)+(-8.8,1.5)+(10,1)$) [anchor=north west]{\includegraphics[scale=0.3]{Cone/Points.png}};}
	\only<3>{\node at ($(current page.south west)+(-8,1.7)+(10,1)$) [anchor=north west]{\includegraphics[scale=0.3]{Cone/Lines.png}};}
	\only<4>{\node at ($(current page.south west)+(-8.6,1.3)+(10,1)$) [anchor=north west]{\includegraphics[scale=0.3]{Cone/Cone.png}};}
\end{tikzpicture} 


\end{frame}

\begin{frame}
\frametitle{Semialgebraic Invariants - Proof Sketch (2)}
\begin{itemize}[<+->]
\item We prove that every semialgebraic invariant must contain the entire cone from some ``height''.
\item Thus, it is enough to check if the cone intersects the halfspace infinitely often.
\item If the cone is semialgebraic - this can be done using decidability of the FO theory of the reals.
\end{itemize}
\includegraphics[scale=0.15]{Cone/Cone.png}
\begin{itemize}[<+->]
\item Problem: the cone might not be semialgebraic.
\item Solution: work with a ``fat'' cone that is semialgebraic.
\end{itemize}
\end{frame}